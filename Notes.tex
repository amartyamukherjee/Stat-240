\documentclass[10pt,letterpaper]{article}
\usepackage[T1]{fontenc}
\usepackage[latin9]{inputenc}
\usepackage{geometry}
\usepackage{amsmath}
\geometry{verbose,tmargin=1.0in,bmargin=1.0in,lmargin=1.0in,rmargin=1.0in}
\usepackage{amssymb}
\usepackage{graphicx}

\begin{document}

\noindent \begin{center}
\textbf{\large{}Stat 240 Notes}\vspace{5mm}
\par\end{center}{\large \par}

\begin{center}
\textbf{1. Foundations}
\end{center}

Probability space: A set of objects $(\Omega, F, p)$:\\

$\Omega$: Sample space\\

$F$: $\sigma $-algebra\\

$p$: Probability\\

\textbf{1.1 $\sigma$-algebra and measure theory}\\

\underline{Definition 1.3: $\sigma$-algebra}\\

$P(\Omega)$ is the power set\\

$F\subseteq P(\Omega)$ is a $\sigma$-algebra on $\Omega$ if:

\begin{enumerate}
\item[i)] $\Omega \in F$
\item[ii)] $A\in F\implies A^c\in F$
\item[iii)] $A, B\in F\implies A\cup B\in F$\\
If (iii) holds for finitely many sets, then $F$ is an algebra on $\Omega$.
\end{enumerate}

\underline{Remark 1.4}\\

$\sigma$-algebras are closed with respect to countable intersections since by De Morgan's Law:
$$\cup_{i=1}^{\infty}A_i = (\cap_{i=1}^{\infty}A_i^c)^c$$

\underline{Definition 1.7: Borel $\sigma$-algebra}\\

$B(\Omega)$ is a Borel $\sigma$-algebra on $\Omega$ and it's elements are Borel sets\\

$B(\Omega) = \sigma (\{A: A\subseteq\Omega, A$ is open$\})$\\

$B(\mathbb{R}^d) = \sigma(\{(\underline{a}, \underline{b}]: \underline{a}\leq \underline{b}\})$\\

\underline{Definition 1.9: Measures}\\

Let $F = \sigma(\Omega)$. $(\Omega, F)$ is a measurable space. And sets in $F$ are measurable sets.\\

A measure $\mu(F)$ is a $\mu$ if:

\begin{enumerate}
\item[i)] $\mu: F\rightarrow [0,\infty]$
\item[ii)] $\mu(\phi) = 0$
\item[iii)] $\sigma$-additivity
\end{enumerate}

$(\Omega, F, \mu)$ is called a measure space.\\

\pagebreak

\textbf{1.2 Probability Measures}\\

\underline{Definition 1.9: Probability measure}\\

Let $(\Omega, F)$ be a measurable space. A probability measure $p$ on $F$ is such that:

\begin{enumerate}
\item[i)] $p: F\rightarrow [0,1]$
\item[ii)] $p(\Omega) = 1$
\item[iii)] $\sigma$-additivity
\end{enumerate}

$(\Omega, F, p)$ is called a probability space.\\

\textbf{1.3 Null sets}\\

Let $(\Omega, F, \mu)$ be a measure space.\\

Every $N\in F$ where $\mu(N)=0$ is a null set\\

\pagebreak

\textbf{1.4 Construction of measures}\\

Idea: Functions with properties as measures (premeasures defined on a ring) can be extended
to complete measures on the $\sigma$-algebra generated by the ring.\\

\underline{Definition 1.18: Ring}\\

$R\in P(\Omega)$ is a ring on $\Omega$ if:

\begin{enumerate}
\item[i)] $\phi\in R$
\item[ii)] $A,B\in R\implies A\textbackslash B\in R$
\item[iii)] $A,B\in R\implies A\cup B\in R$
\end{enumerate}

A \underline{premeasure} $\mu_0$ on $R$ is a function such that:

\begin{enumerate}
\item[i)] $\mu_0: R\rightarrow [0,\infty]$
\item[ii)] $\mu(\phi) = 0$
\item[iii)] $\sigma$-additivity
\end{enumerate}

\underline{Theorem 1.19: Caratheorody's exclusion theorem}\\

Let $\mu_0$ be a premeasure to $R$ on $\Omega$. There exists a complete measure $\mu$ on $F:\sigma(R)$
which coincides with $\mu_0$ on $R$.\\

If $\mu_0$ is $\sigma$-finite, then $\mu$ is unique.\\

\underline{Theorem 1.21}\\

$F:R\rightarrow R$ right-continuous and increasing. There exists a unique Borel measure $\mu_F$
such that $\mu_F([a,b])=F(b)-F(a)$.\\

\underline{Remark 1.22}\\

\begin{enumerate}
\item[1)] By Theorem 1.19, $\mu_F$ is complete and is called the 
\underline{Lebesgue-Stielties measure} associated to $F$. It's domain $\overline{B(R)}$ known as
\underline{Lebesgue $\sigma$-algebra} can be shown to strictly contain $B(R)$. Sets in 
$\overline{B(R)}$ are \underline{Lebesgue measurable} (or \underline{Lebesgue sets})
\item[2)] If $F(x) = x$, $\Lambda :=\mu_F$ is called \underline{Lebesgue measure on $R$}
and sets $N\in \overline{B(R)}: \Lambda(N)=0$ \underline{Lebesgue null sets}.
\end{enumerate}

\underline{Remark 1.24}\\

\begin{enumerate}
\item[1)] Theorem 1.21 extends to $F:R^d\rightarrow R$ which is:
  \item[i)] Right-continuous: $F(\underline{x}) = {lim}_{\underline{h}\rightarrow\underline{0}}
  F(\underline{x}+\underline{h})$
  \item[ii)] d-increasing: The $F$-volume $\Delta_{(\underline{a},\underline{b}]}F$ of 
  $(\underline{a},\underline{b}]$ is $\geq 0$ for all $\underline{a}\leq\underline{b}$, where:\\
  $$\Delta_{[\underline{a},\underline{b}]}F = \Pi_{i=1}^d(b_j-a_j)$$
  \begin{center}e.g. $d=2$, $\underline{a}=(a_1, a_2), \underline{b}=(b_1, b_2)$\end{center}
  $$\Delta_{[\underline{a},\underline{b}]}F = F(b_1, b_2) - F(a_1, b_2) - F(b_1, a_2) + F(a_1, a_2)$$
\item[2)] If ${lim}_{x_j\rightarrow -\infty} F(\underline{x})=0$ for some $j\in\{1,\dots,d\}$ and
$F(\infty)={lim}_{\underline{x}\rightarrow -\infty} F(\underline{x})=1$, then $\mu_F$ is
a probability measure on $B(R^d)$.

\end{enumerate}
\pagebreak

\begin{center}
\textbf{2. Geometric and Laplace Probability}
\end{center}

\underline{Prop 2.1}\\

Let $(\Omega, F, \mu)$ be a measure space, where $0<\mu(\Omega)<\infty$\\

Then $(\Omega, F, p)$ with $p(A)=\frac{\mu(A)}{\mu(\Omega)}$ is a probability space.\\

$F$ is a $\sigma$-algebra on $\Omega$ and $\Omega' \subseteq \Omega$, then one can show: The
restriction $F|_{\Omega}:=\{A\cup \Omega': A\in F\}$ is a $\sigma$-algebra on $\Omega'$.\\

\underline{Ref 2.2}\\

$\Omega\subseteq R^d: 0<\Lambda(\Omega)<\infty, F=\overline{B}(\Omega), 
p(A)=\frac{\mu(A)}{\mu(\Omega)}$, for all $A\in F$, then the probability space $(\Omega, F, P)$ is 
called \underline{geometric probability space}.

\underline{Prop 2.4}\\

$1\leq |\Omega| <\infty$, $F=P(\Omega)$, $p(A) = \frac{|A|}{|\Omega|}$\\

Then $(\Omega, F, P)$ is a finite probabiity space called \underline{Laplace probability space}.\\

P is called \underline{discrete uniform distribution} on $\Omega$.









\end{document}